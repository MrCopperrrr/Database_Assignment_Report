\section*{3. Hiện thực ứng dụng}

Nhóm đã xây dựng ứng dụng web Shopmie (mô phỏng Shopee) để minh họa việc kết nối và thao tác với cơ sở dữ liệu MySQL. Ứng dụng bao gồm hai phân hệ chính: Kênh Người Bán (Quản lý dữ liệu) và Cửa Hàng (Hiển thị và tương tác người dùng).

\subsection*{3.1. Chức năng quản lý dữ liệu (Thêm/Xóa/Sửa)}

Giao diện "Kênh Người Bán" được hiện thực hóa để cho phép người dùng (với vai trò Seller) thao tác trực tiếp với bảng Product và các bảng liên quan (product\_image, product\_video, product\_variant) trong cơ sở dữ liệu.  
Tại màn hình chính, danh sách sản phẩm được hiển thị dưới dạng bảng, cung cấp cái nhìn tổng quan về: Tên sản phẩm, Giá gốc, Giá khuyến mãi, Tồn kho và Trạng thái (Available, Out of Stock, v.v.).
\begin{center}
    \includegraphics[scale=0.8]{Picture/127.png}
\end{center}
\textbf{Hình 1:} Hình ảnh kênh người bán

\textbf{Chức năng Thêm mới (Create):}  
Khi người dùng nhấn nút "Thêm Sản Phẩm", một cửa sổ (Modal) sẽ hiện ra. Hệ thống yêu cầu người dùng nhập đầy đủ các thông tin như: Hình ảnh/Video (URL), Tên sản phẩm, Danh mục, Giá, Kho và Mô tả.
\begin{center}
    \includegraphics[scale=0.8]{Picture/128.png}
\end{center}
\textbf{Hình 2:} Thêm dữ liệu hàng hóa Smart Tivi LG Evo Oled 4K 55 inch

\textbf{Chức năng Cập nhật (Update):}  
Khi nhấn nút "Sửa" trên một dòng sản phẩm, hệ thống sẽ gọi API lấy chi tiết dữ liệu hiện tại từ Database và điền vào Form (Pre-fill). Người dùng có thể chỉnh sửa lại giá bán, cập nhật số lượng tồn kho hoặc thay đổi thông tin mô tả.
\begin{center}
    \includegraphics[scale=0.8]{Picture/129.png}
\end{center}
\textbf{Hình 3:} Chỉnh sửa giá và tồn kho của Smart Tivi LG Evo Oled 4K 55 inch

\textbf{Chức năng Xóa (Delete) và Xử lý lỗi logic:}  
Để đảm bảo an toàn dữ liệu, khi người dùng chọn "Xóa", hệ thống sẽ hiển thị một thông báo xác nhận (Confirm Dialog) để tránh thao tác nhầm lẫn.  
Xử lý Logic: Trước khi thực hiện lệnh DELETE trong CSDL, Backend sẽ kiểm tra các ràng buộc khóa ngoại (Foreign Key). Nếu sản phẩm đã nằm trong một đơn hàng (ORDER\_LINE), hệ thống sẽ chặn việc xóa và thông báo lỗi phù hợp để bảo toàn lịch sử giao dịch. Nếu sản phẩm chưa có phát sinh giao dịch, việc xóa sẽ được thực thi thành công.
\begin{center}
    \includegraphics[scale=0.8]{Picture/130.png}
\end{center}
\textbf{Hình 4:} Xác nhận trước khi xóa sản phẩm

Sau khi xóa thành công, danh sách sẽ tự động được tải lại (Re-fetch) để cập nhật giao diện mới nhất mà không cần tải lại trang.
\begin{center}
    \includegraphics[scale=0.8]{Picture/131.png}
\end{center}
\textbf{Hình 5:} Sau khi xóa dữ liệu thành công

\subsection*{3.2. Giao diện hiển thị, tìm kiếm và tương tác người dùng}

Phần giao diện "Cửa hàng" (Storefront) được xây dựng để minh họa cho việc gọi các Thủ tục lưu trữ (Stored Procedures) đã viết ở câu 2.3 (ví dụ: sp\_SearchProductByCategoryAndPrice). Giao diện này tập trung vào trải nghiệm người dùng với bố cục lưới (Grid layout) trực quan.  
Màn hình chính hiển thị danh sách sản phẩm với đầy đủ thông tin cần thiết cho người mua: Hình ảnh, Tên, Giá bán, Nhãn giảm giá (Discount Badge) và Số lượng đã bán.
\begin{center}
    \includegraphics[scale=0.8]{Picture/132.png}
\end{center}
\textbf{Hình 6:} Giao diện cửa hàng hiển thị các sản phẩm

\textbf{Chức năng Tìm kiếm và Lọc (Search \& Filter):}  
Hệ thống cung cấp thanh điều hướng theo danh mục (Tabs). Khi người dùng chọn một danh mục (ví dụ: "Điện thoại"), ứng dụng sẽ gửi tham số CategoryName xuống Database thông qua thủ tục lưu trữ để lọc ra chính xác các sản phẩm thuộc nhóm đó.
\begin{center}
    \includegraphics[scale=0.8]{Picture/133.png}
\end{center}
\textbf{Hình 7:} Lọc theo danh mục sản phẩm Điện thoại

\textbf{Chức năng Sắp xếp (Sorting):}  
Người dùng có thể sắp xếp danh sách sản phẩm theo các tiêu chí: Phổ biến, Mới nhất, Bán chạy và Giá (Thấp đến Cao / Cao đến Thấp). Việc này giúp người dùng dễ dàng tìm kiếm sản phẩm phù hợp với nhu cầu và ngân sách.
\begin{center}
    \includegraphics[scale=0.8]{Picture/134.png}
\end{center}
\textbf{Hình 8:} Sắp xếp giá từ cao đến thấp

\textbf{Chi tiết sản phẩm và Validate logic mua hàng:}  
Khi nhấp vào một sản phẩm, Modal chi tiết sẽ hiện ra với đầy đủ mô tả và thông số kỹ thuật. Tại đây, hệ thống hiển thị số lượng tồn kho thực tế.
\begin{center}
    \includegraphics[scale=0.8]{Picture/135.png}
\end{center}
\textbf{Hình 9:} Hiển thị thông tin chi tiết khi chọn sản phẩm

\textbf{Chức năng Thêm vào giỏ hàng:}  
Khi người dùng nhấn "Thêm Vào Giỏ Hàng", hệ thống sẽ thực hiện kiểm tra logic (sử dụng Hàm 2.4 \texttt{fn\_ValidateCartStock}).
\begin{center}
    \includegraphics[scale=0.8]{Picture/136.png}
\end{center}
\textbf{Hình 10:} Chức năng thêm vào giỏ hàng

\subsection*{3.3. Giao diện thủ tục khác}

\paragraph*{Tổng quan}
Hàm \texttt{fn\_CalculateCartTotal} được thiết kế để phục vụ trải nghiệm mua sắm của người dùng tại bước xem Giỏ hàng. Chức năng này thực hiện việc quét qua toàn bộ các sản phẩm mà người dùng đã thêm vào giỏ, tính toán tổng số tiền cần thanh toán dựa trên đơn giá hiện tại và số lượng đã chọn. Kết quả trả về là con số Tổng số tiền phải trả giúp người dùng nắm bắt được giá trị đơn hàng trước khi tiến hành đặt mua.

\paragraph*{Luồng xử lí}
Quy trình xử lý dữ liệu bắt đầu từ giao diện "Giỏ hàng/Thanh toán".
\begin{center}
    \includegraphics[scale=0.8]{Picture/137.png}
\end{center}
Hàm sẽ xác định StoreID dựa trên tên shop "John Electronics Store".

Hàm sẽ lấy dữ liệu từ cột Đơn giá và ô Số lượng của từng sản phẩm trong ảnh để thực hiện phép nhân và cộng dồn.
\begin{center}
    \includegraphics[scale=0.8]{Picture/138.png}
\end{center}
Gồm có danh sách hàng hoá hiện có trong giỏ hàng và các thông tin chi tiết về món hàng như là giá tiền, số lượng và tổng số tiền cuối cùng phải thanh toán.

Số tiền 42.180.000đ trong ảnh là kết quả trả về của hàm \texttt{fn\_CalculateCartTotal} với tham số đầu vào là \texttt{p\_CartID}. Hàm sẽ thực hiện phép tính: Giá tiền món 1 * số lượng món 1 + Giá tiền món 2 * số lượng món 2 + … 

Và kết quả cuối cùng trả về là giá trị của giỏ hàng này.